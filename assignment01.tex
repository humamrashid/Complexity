% Assignment #1
% Humam Rashid
% Spring 2020, CISC 7214X, Prof. Cox

\documentclass{scrartcl}
\usepackage{amsmath, amsthm}
\usepackage[a4paper, margin=1in]{geometry}
\usepackage[shortlabels]{enumitem}
\begin{document}
\begin{flushleft}

    Humam Rashid\\
    Assignment 1\\
    CISC 7214X, Prof. Cox\\
    \bigskip
    
    1. Prove the following identity for any positive $n \geq 1$:
    \begin{align*}
        \sum_{i=1}^{n}\sum_{j=1}^{i} 2 = n (n + 1)
    \end{align*}

    \begin{proof}
        Let $n = 1,2,3$ in succession, then:
        \begin{align}
            p(1) &= \sum_{i=1}^{1}\sum_{j=1}^{i} 2 = \sum_{j=1}^{1} 2 = 2 = 1 (1 + 1)\\
            p(2) &= \sum_{i=1}^{2}\sum_{j=1}^{i} 2 = \sum_{j=1}^{1} 2
            + \sum_{j=1}^{2} 2 = 6 = 2 (2 + 1)\\
            p(3) &= \sum_{i=1}^{3}\sum_{j=1}^{i} 2 = \sum_{j=1}^{1} 2
            + \sum_{j=1}^{2} 2 + \sum_{j=1}^{3} 2 = 12 = 3 (3 + 1)
        \end{align}
        Thus, we can assume:
        \begin{align}
            p(k) = \sum_{i=1}^{k}\sum_{j=1}^{i} 2 = k (k + 1) = k^2 + k
        \end{align}
        We want to show that:
        \begin{align*}
            p(k + 1) &= \sum_{i=1}^{k + 1}\sum_{j=1}^{i} 2 \\
                     &= (k + 1) ((k + 1) + 1) \\
                     &= (k + 1) (k + 2) \\
                     &= k^2 + 3k + 2
        \end{align*}
        Then, for $k \geq 1$, we have:
        \begin{align}
            p(k + 1) = \sum_{i=1}^{k + 1}\sum_{j=1}^{i} 2 = \sum_{j=1}^{1} 2 + \ldots
            + \sum_{j=1}^{k - 1} 2 + \sum_{j=1}^{k} 2 + \sum_{j=1}^{k+1} 2
        \end{align}
    \end{proof}

\end{flushleft}
\end{document}

% EOF.
