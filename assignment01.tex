% Assignment #1
% Humam Rashid
% Spring 2020, CISC 7214X, Prof. Cox

\documentclass{scrartcl}
\usepackage{amsmath, amsthm}
\usepackage[a4paper, margin=1in]{geometry}
\usepackage[shortlabels]{enumitem}
\begin{document}
\begin{flushleft}

    Humam Rashid\\
    Assignment 1\\
    CISC 7214X, Prof. Cox\\
    \bigskip
    
    1. Prove the following identity for any positive $n \geq 1$:
    \begin{align*}
        p(n) = \sum_{i=1}^{n}\sum_{j=1}^{i} 2 = n (n + 1)
    \end{align*}
    \begin{proof}
        Let $n = 1,2,3$ in succession. Using the distributive identity of summation, we have:
        \begin{align*}
            p(1) &= \sum_{i=1}^{1}\sum_{j=1}^{i} 2 = \sum_{i=1}^{1} 2(i) = 2(1) = 2 = 1(1 + 1)\\
            p(2) &= \sum_{i=1}^{2}\sum_{j=1}^{i} 2 = \sum_{i=1}^{2} 2(i) = 2(1) + 2(2) = 6
            = 2(2 + 1)\\
            p(3) &= \sum_{i=1}^{3}\sum_{j=1}^{i} 2 = \sum_{i=1}^{3} 2(i) = 2(1) + 2(2) + 2(3) = 12
            = 3(3 + 1)
            \tag{1}
        \end{align*}
        Thus, we can assume:
        \begin{align*}
            p(k) = \sum_{i=1}^{k}\sum_{j=1}^{i} 2 = \sum_{i=1}^{k} 2(i) = k(k + 1) = k^2 + k
            \tag{2}
        \end{align*}
        We want to show that:
        \begin{align*}
            p(k + 1) = \sum_{i=1}^{k + 1}\sum_{j=1}^{i} 2 = (k + 1) ((k + 1) + 1) = (k + 1) (k + 2) = k^2 + 3k + 2
        \end{align*}
        For $k \geq 1$, we have:
        \begin{align*}
            p(k + 1) = \sum_{i=1}^{k + 1}\sum_{j=1}^{i} 2 &= \sum_{i=1}^{k+1} 2(i) = 2(1) + \ldots
            + 2(k - 1) + 2(k) + 2(k + 1) \\
            &= \sum_{i=1}^{k} 2(i) + 2(k + 1) = k^2 + k + 2k + 2 = k^2 + 3k + 2 \\
            &= (k + 1) ((k + 1) + 1)
            \tag{3}
        \end{align*}
    \end{proof}
    \newpage
    2. Prove the following identity for any positive $n \geq 1$:

\end{flushleft}
\end{document}

% EOF.
