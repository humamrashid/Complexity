% Assignment #1
% Humam Rashid
% Spring 2020, CISC 7214X, Prof. Cox

\documentclass{scrartcl}
\usepackage{amsmath, amsthm, amssymb}
\usepackage[a4paper, margin=1in]{geometry}
\begin{document}
\begin{flushleft}

    Humam Rashid\\
    Assignment 1\\
    CISC 7214X, Prof. Cox\\
    \bigskip
    
    %%%%%%%%%%%%%%%%%%%%%%%%%%%%%%%%%%%%%%%%%%%%%%%%%%%%%%%%%%%%%%%%%%%%%%%%%%%%%%%%%%%%%%%%%%%%%%%%
    % Problem 1 on proof by induction.
    %%%%%%%%%%%%%%%%%%%%%%%%%%%%%%%%%%%%%%%%%%%%%%%%%%%%%%%%%%%%%%%%%%%%%%%%%%%%%%%%%%%%%%%%%%%%%%%%
    1. Prove the following identity for any positive $n \geq 1$:
    \begin{align*}
        p(n) = \sum_{i=1}^{n}\sum_{j=1}^{i} 2 = n (n + 1)
    \end{align*}
    \begin{proof}
        Let $n = 1,2,3$ in succession. Using the distributive identity of summation, we have:
        \begin{align*}
            p(1) &= \sum_{i=1}^{1}\sum_{j=1}^{i} 2 = \sum_{i=1}^{1} 2(i) = 2(1) = 2 = 1(1 + 1)\\
            p(2) &= \sum_{i=1}^{2}\sum_{j=1}^{i} 2 = \sum_{i=1}^{2} 2(i) = 2(1) + 2(2) = 6
            = 2(2 + 1)\\
            p(3) &= \sum_{i=1}^{3}\sum_{j=1}^{i} 2 = \sum_{i=1}^{3} 2(i) = 2(1) + 2(2) + 2(3) = 12
            = 3(3 + 1)
        \end{align*}
        Thus, we can assume:
        \begin{align*}
            p(k) = \sum_{i=1}^{k}\sum_{j=1}^{i} 2 = \sum_{i=1}^{k} 2(i) = k(k + 1) = k^2 + k
        \end{align*}
        We want to show that:
        \begin{align*}
            p(k + 1) = \sum_{i=1}^{k + 1}\sum_{j=1}^{i} 2 = (k + 1) ((k + 1) + 1) = (k + 1) (k + 2)
            = k^2 + 3k + 2
        \end{align*}
        For $k \geq 1$, we have:
        \begin{align*}
            p(k + 1) = \sum_{i=1}^{k + 1}\sum_{j=1}^{i} 2 &= \sum_{i=1}^{k+1} 2(i) = 2(1) + \ldots
            + 2(k - 1) + 2(k) + 2(k + 1) \\
            &= \sum_{i=1}^{k} 2(i) + 2(k + 1) = k^2 + k + 2k + 2 = k^2 + 3k + 2 \\
            &= (k + 1) ((k + 1) + 1)
        \end{align*}
    \end{proof}
    \newpage
    %%%%%%%%%%%%%%%%%%%%%%%%%%%%%%%%%%%%%%%%%%%%%%%%%%%%%%%%%%%%%%%%%%%%%%%%%%%%%%%%%%%%%%%%%%%%%%%%
    % Problem 2 on recurrence equations.
    %%%%%%%%%%%%%%%%%%%%%%%%%%%%%%%%%%%%%%%%%%%%%%%%%%%%%%%%%%%%%%%%%%%%%%%%%%%%%%%%%%%%%%%%%%%%%%%%
    2. Solve the following recurrence and prove the solution correct with induction:
    \begin{align*}
        T(1) &= 2c\\
        T(n) &= 2cn + T(\frac{n}{2})
    \end{align*}
    Some small evaluations for the recurrence are:
    \medskip
    \begin{center}
        \begin{tabular}{|c|c|c|c|c|c|}
            \hline
            $n$ & $1$ & $2$ & $4$ & $8$ & $16$\\
            \hline
            $T_n$ & $2c$ & $6c$ & $14c$ & $30c$ & $62c$\\
            \hline
        \end{tabular}
    \end{center}
    \medskip
    \textit{Solution}. Our guess is that $T(n) = \Theta(n)$, but more specifically:
    \begin{equation}
        T(n) \leq 4cn\ \text{for some}\ c > 0
        \label{eq:1}
    \end{equation}
    We prove that $T(n) \leq 4cn$ assuming that an appropriate choice is made for the constant $c >
    0$, that $n$ is a power of $2$, and that equation~\ref{eq:1} holds for all positive $m < n$
    (in particular for $m = \frac{n}{2} \implies T(\frac{n}{2}) \leq 4\frac{cn}{2}$):
    \begin{align*}
        T(n) &= 2cn + T(\frac{n}{2})\\
        &\leq 2cn + 4\frac{cn}{2}\\
        &= 2cn + 2cn\\
        &= 4cn
    \end{align*}
    \begin{proof}
    \end{proof}
    \newpage
    %%%%%%%%%%%%%%%%%%%%%%%%%%%%%%%%%%%%%%%%%%%%%%%%%%%%%%%%%%%%%%%%%%%%%%%%%%%%%%%%%%%%%%%%%%%%%%%%
    % Problem 3 on efficient selection.
    %%%%%%%%%%%%%%%%%%%%%%%%%%%%%%%%%%%%%%%%%%%%%%%%%%%%%%%%%%%%%%%%%%%%%%%%%%%%%%%%%%%%%%%%%%%%%%%%
    3. Given a large unsorted array of salaries, give an efficient algorithm and its average case
    complexity to print the top $50\%$ of the salaries. Hint: can you do it without sorting?
    \begin{proof}[Selection Algorithm]\let\qed\relax
        One algorithm that can be used to solve this problem is \textit{Quickselect}, a selection
        algorithm which uses the partitioning technique from Quicksort. Pseudo-code for Quickselect
        based on McConnell's version in \textit{Analysis of Algorithms} is shown below:
        \medskip
\begin{verbatim}
SelectKthLargest(array, start, end, K) {
    if start < end {
        RandomizedPartition(array, start, end, pivot)
        if pivot == K {
            return array[pivot]
        } else {
            if K < pivot {
                return SelectKthLargest(
                    array, pivot + 1, end, K)
            } else {
                return SelectKthLargest(
                    array, start, pivot - 1, K - pivot)
            }
        }
    }
}
\end{verbatim}
        where \verb|array| is the unsorted array of salary values, \verb|start| is the index of the
        first value in the array, \verb|end| is the index of the last value to consider in the
        array, and \verb|K| is the element in this array that is the $K$th largest value. The
        \verb|RandomizedPartition()| procedure works similarly to what one expects to find in a
        Quicksort algorithm, but picks a random pivot rather than a fixed one. Quickselect has an
        average case complexity of $O(n)$ and a worst case analysis of $O(n^2)$. The expected linear
        time algorithm is strengthened by the randomized pivot so worst case behavior is unlikely.
        \bigskip

        For this particular problem, the \verb|SelectKthLargest()| procedure would leave the array
        with two parts, those smaller than the $K$th value and those larger than it. The array would
        be rearranged so that that smaller values are before it and the larger ones after. This
        means to get the top $m$ values, we can use a simple loop (after running our Quickselect
        algorithm) to retrieve them. Therefore, retrieving the top $m$ values requires $O(n + m)$
        time. For the top $50\%$, $m = \frac{n}{2}$, meaning the complexity would still simplify to
        $O(n)$ expected time.
    \end{proof}
    \medskip
    \begin{proof}[Heap-Based Alternative]\let\qed\relax
        An alternative solution that does not require any sorting is to build a max-heap from the
        elements of the unsorted array in $O(n)$ time. Then, we can extract the top $K$ elements in
        $O(K * lg(K))$ time, resulting in $O(n + K * lg(K))$. In our problem, $K = \frac{n}{2}$.
        Hence, the complexity of this algorithm to get the top $50\%$ becomes $O(n + \frac{n}{2} *
        lg(\frac{n}{2}))$ or $ O(n + n (\frac{lg(n)}{2} - \frac{1}{2}))$.
    \end{proof}

    \newpage
    %%%%%%%%%%%%%%%%%%%%%%%%%%%%%%%%%%%%%%%%%%%%%%%%%%%%%%%%%%%%%%%%%%%%%%%%%%%%%%%%%%%%%%%%%%%%%%%%
    % Problem 4 on graph algorithms.
    %%%%%%%%%%%%%%%%%%%%%%%%%%%%%%%%%%%%%%%%%%%%%%%%%%%%%%%%%%%%%%%%%%%%%%%%%%%%%%%%%%%%%%%%%%%%%%%%
    4. Suppose that you are given $n$ cities and the distance between each pair of cities. You
    desire to be able to send a message between any pair of cities. Give an algorithm to lay the
    least amount of two-way cable so that a network that connects the cities is formed.
    \begin{proof}[Minimum Spanning Tree]\let\qed\relax
        To solve this problem, we consider the set of $n$ cities as vertices on a graph and the
        distance between any pair of cities as its edges. Thereafter, we use a minimum spanning tree
        (MST) algorithm, which gets us a subgraph with the complete set of vertices (cities) as in
        the original graph but a potentially smaller subset of edges (distances) with minimal weight
        sum. Very high-level pseudo-code for Kruskal's MST algorithm, based on Horowitz and Sahni's
        in \textit{Fundamentals of Computer Algorithms}, is shown below:
        \medskip
\begin{verbatim}
Kruskal-MST(E, n, T) {
    T = null
    while T contains less than n - 1 edges {
        choose an edge (v, w) from E of lowest weight
        delete edge (v, w) from E
        if edge (v, w) does not create a cycle in T {
            add edge (v, w) to T
        } else {
            discard edge (v, w)
        }
    }
}
\end{verbatim}
        where \verb|E| is the set of edges in the connected graph representing the map of \verb|n|
        cities and \verb|T| is the set of edges in the MST.
        \bigskip

        Given the connected graph $G(E, V)$, where $V$ is the set of vertices and $E$ is the set of
        edges, Kruskal's MST algorithm has complexity $O(E * lg(E))$ or equivalently, $O(E *
        lg(V))$.
    \end{proof}

\end{flushleft}
\end{document}

% EOF.
