% Assignment #1
% Humam Rashid
% Spring 2020, CISC 7214X, Prof. Cox

\documentclass{scrartcl}
\usepackage{amsmath, amsthm}
\usepackage[a4paper, margin=1in]{geometry}
\begin{document}
\begin{flushleft}

    Humam Rashid\\
    Assignment 1\\
    CISC 7214X, Prof. Cox\\
    \bigskip
    
    %%%%%%%%%%%%%%%%%%%%%%%%%%%%%%%%%%%%%%%%%%%%%%%%%%%%%%%%%%%%%%%%%%%%%%%%%%%%%%%%%%%%%%%%%%%%%%%%
    % Problem 1 on proof by induction.
    %%%%%%%%%%%%%%%%%%%%%%%%%%%%%%%%%%%%%%%%%%%%%%%%%%%%%%%%%%%%%%%%%%%%%%%%%%%%%%%%%%%%%%%%%%%%%%%%
    1. Prove the following identity for any positive $n \geq 1$:
    \begin{align*}
        p(n) = \sum_{i=1}^{n}\sum_{j=1}^{i} 2 = n (n + 1)
    \end{align*}
    \begin{proof}
        Let $n = 1,2,3$ in succession. Using the distributive identity of summation, we have:
        \begin{align*}
            p(1) &= \sum_{i=1}^{1}\sum_{j=1}^{i} 2 = \sum_{i=1}^{1} 2(i) = 2(1) = 2 = 1(1 + 1)\\
            p(2) &= \sum_{i=1}^{2}\sum_{j=1}^{i} 2 = \sum_{i=1}^{2} 2(i) = 2(1) + 2(2) = 6
            = 2(2 + 1)\\
            p(3) &= \sum_{i=1}^{3}\sum_{j=1}^{i} 2 = \sum_{i=1}^{3} 2(i) = 2(1) + 2(2) + 2(3) = 12
            = 3(3 + 1)
        \end{align*}
        Thus, we can assume:
        \begin{align*}
            p(k) = \sum_{i=1}^{k}\sum_{j=1}^{i} 2 = \sum_{i=1}^{k} 2(i) = k(k + 1) = k^2 + k
        \end{align*}
        We want to show that:
        \begin{align*}
            p(k + 1) = \sum_{i=1}^{k + 1}\sum_{j=1}^{i} 2 = (k + 1) ((k + 1) + 1) = (k + 1) (k + 2)
            = k^2 + 3k + 2
        \end{align*}
        For $k \geq 1$, we have:
        \begin{align*}
            p(k + 1) = \sum_{i=1}^{k + 1}\sum_{j=1}^{i} 2 &= \sum_{i=1}^{k+1} 2(i) = 2(1) + \ldots
            + 2(k - 1) + 2(k) + 2(k + 1) \\
            &= \sum_{i=1}^{k} 2(i) + 2(k + 1) = k^2 + k + 2k + 2 = k^2 + 3k + 2 \\
            &= (k + 1) ((k + 1) + 1)
        \end{align*}
    \end{proof}
    \newpage
    %%%%%%%%%%%%%%%%%%%%%%%%%%%%%%%%%%%%%%%%%%%%%%%%%%%%%%%%%%%%%%%%%%%%%%%%%%%%%%%%%%%%%%%%%%%%%%%%
    % Problem 2 on recurrence equations.
    %%%%%%%%%%%%%%%%%%%%%%%%%%%%%%%%%%%%%%%%%%%%%%%%%%%%%%%%%%%%%%%%%%%%%%%%%%%%%%%%%%%%%%%%%%%%%%%%
    2. Solve the following recurrence and prove the solution correct with induction:
    \begin{align*}
        T(1) &= 2c\\
        T(n) &= 2cn + T(\frac{n}{2})
    \end{align*}
    Some small evaluations for the recurrence are:
    \medskip
    \begin{center}
        \begin{tabular}{|c|c|c|c|c|c|}
            \hline
            $n$ & $1$ & $2$ & $4$ & $8$ & $16$\\
            \hline
            $T_n$ & $2c$ & $6c$ & $14c$ & $30c$ & $62c$\\
            \hline
        \end{tabular}
    \end{center}
    \medskip
    \textit{Solution}. Our guess is that $T(n) = \Theta(n)$, but more specifically:
    \begin{equation}
        T(n) \leq 4cn\ \text{for some}r c > 0
        \label{eq:1}
    \end{equation}
    We prove that $T(n) \leq 4cn$ assuming that an appropriate choice is made for the constant $c >
    0$, that $n$ is a power of $2$, and that equation~\ref{eq:1} holds for all positive $m < n$
    (particularly for $m = \frac{n}{2} \implies T(\frac{n}{2}) \leq 4\frac{cn}{2}$):
    \begin{align*}
        T(n) &= 2cn + T(\frac{n}{2})\\
        &\leq 2cn + 4\frac{cn}{2}\\
        &= 2cn + 2cn\\
        &= 4cn
    \end{align*}
    \begin{proof}
    \end{proof}
    \newpage
    %%%%%%%%%%%%%%%%%%%%%%%%%%%%%%%%%%%%%%%%%%%%%%%%%%%%%%%%%%%%%%%%%%%%%%%%%%%%%%%%%%%%%%%%%%%%%%%%
    % Problem 3 on efficient selection.
    %%%%%%%%%%%%%%%%%%%%%%%%%%%%%%%%%%%%%%%%%%%%%%%%%%%%%%%%%%%%%%%%%%%%%%%%%%%%%%%%%%%%%%%%%%%%%%%%
    3. Given a large unsorted array of salaries, give an efficient algorithm and its average case
    complexity to print the top $50\%$ of the salaries. Hint: can you do it without sorting?
    \newpage
    %%%%%%%%%%%%%%%%%%%%%%%%%%%%%%%%%%%%%%%%%%%%%%%%%%%%%%%%%%%%%%%%%%%%%%%%%%%%%%%%%%%%%%%%%%%%%%%%
    % Problem 4 on graph algorithms.
    %%%%%%%%%%%%%%%%%%%%%%%%%%%%%%%%%%%%%%%%%%%%%%%%%%%%%%%%%%%%%%%%%%%%%%%%%%%%%%%%%%%%%%%%%%%%%%%%
    4. Suppose that you are given $n$ cities and the distance between each pair of cities. You
    desire to be able to send a message between any pair of cities. Give an algorithm to lay the
    least amount of two-way cable so that a network that connects the cities is formed.

\end{flushleft}
\end{document}

% EOF.
