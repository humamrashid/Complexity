% Assignment #2
% Humam Rashid
% Spring 2020, CISC 7214X, Prof. Cox

\documentclass{scrartcl}
\usepackage{amsmath, amsthm, amssymb}
\usepackage[a4paper, margin=1in]{geometry}
\usepackage[shortlabels]{enumitem}
\begin{document}
\begin{flushleft}

    Humam Rashid\\
    Assignment 2\\
    CISC 7214X, Prof. Cox\\
    \bigskip

    %%%%%%%%%%%%%%%%%%%%%%%%%%%%%%%%%%%%%%%%%%%%%%%%%%%%%%%%%%%%%%%%%%%%%%%%%%%%%%%%%%%%%%%%%%%%%%%%
    % Problem 1.
    %%%%%%%%%%%%%%%%%%%%%%%%%%%%%%%%%%%%%%%%%%%%%%%%%%%%%%%%%%%%%%%%%%%%%%%%%%%%%%%%%%%%%%%%%%%%%%%%
    1. For each of the following statements, write if it is ``True,'' ``False,'' or ``We don't
    know.'' Justify your answers.\\
    \begin{enumerate}[(a)]
        \item A problem $Q$ can be in $\mathcal{P}$ and $\mathcal{NP}$.\\
            \medskip
            \textbf{True}. According to the definitions of $\mathcal{P}$ and $\mathcal{NP}$ given in
            ~\cite{modernapp}, $\mathcal{P} \subseteq \mathcal{NP}$. Therefore, if $Q \in
            \mathcal{P}$, then $Q \in \mathcal{NP}$. This also follows from the intuitive notion
            that $\mathcal{P}$ is the class of problems with ``efficiently computable solutions''
            and $\mathcal{NP}$ is the class of problems with ``efficiently verifiable solutions'',
            i.e., any solutions that are efficiently solvable must necessarily also be efficiently
            verifiable. The reverse, whether $\mathcal{NP} \subseteq \mathcal{P}$, is not currently
            known.
        \item A problem $Q$ can be in $\mathcal{P}$ and $\mathcal{NP}$-Complete.\\
            \medskip
            \textbf{We don't know}.
        \item A problem $Q$ can be in $\mathcal{NP}$ and $\mathcal{NP}$-Complete.\\
            \medskip
            \textbf{True}. This is directly from the definition of $\mathcal{NP}$-Complete.
    \end{enumerate}
    %%%%%%%%%%%%%%%%%%%%%%%%%%%%%%%%%%%%%%%%%%%%%%%%%%%%%%%%%%%%%%%%%%%%%%%%%%%%%%%%%%%%%%%%%%%%%%%%
    % Problem 2.
    %%%%%%%%%%%%%%%%%%%%%%%%%%%%%%%%%%%%%%%%%%%%%%%%%%%%%%%%%%%%%%%%%%%%%%%%%%%%%%%%%%%%%%%%%%%%%%%%
    2. Let $X$ be a problem that is in the class $\mathcal{NP}$. For each of the following
    statements, write if it is ``True,'' ``False,'' or ``We don't know.'' Justify your answers.\\
    \begin{enumerate}[(a)]
        \item Solutions to $X$ can be verified with a polynomial-time algorithm.\\
            \medskip
            \textbf{True}. This is directly from the definition of $\mathcal{NP}$.
        \item There is no polynomial-time algorithm for $X$.\\
            \medskip
            \textbf{We don't know}.
        \item If there exists a polynomial-time algorithm for $X$, then $\mathcal{P} =
            \mathcal{NP}$.
        \item If there exists a polynomial-time algorithm for $X$, then there must exist a
            polynomial-time algorithm for the Traveling Salesman Problem.
        \item If there exists a polynomial-time algorithm for the Traveling Salesman Problem, then
            there must exist a polynomial-time algorithm for $X$.
    \end{enumerate}

    \bibliography{complexity}
    \bibliographystyle{plain}
\end{flushleft}
\end{document}

% EOF.
