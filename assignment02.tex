% Assignment #2
% Humam Rashid
% Spring 2020, CISC 7214X, Prof. Cox

\documentclass{scrartcl}
\usepackage{amsmath, amsthm, amssymb}
\usepackage[a4paper, margin=1in]{geometry}
\usepackage[shortlabels]{enumitem}
\begin{document}
\begin{flushleft}

    Humam Rashid\\
    Assignment 2\\
    CISC 7214X, Prof. Cox\\
    \bigskip

    %%%%%%%%%%%%%%%%%%%%%%%%%%%%%%%%%%%%%%%%%%%%%%%%%%%%%%%%%%%%%%%%%%%%%%%%%%%%%%%%%%%%%%%%%%%%%%%%
    % Problem 1.
    %%%%%%%%%%%%%%%%%%%%%%%%%%%%%%%%%%%%%%%%%%%%%%%%%%%%%%%%%%%%%%%%%%%%%%%%%%%%%%%%%%%%%%%%%%%%%%%%
    1. \textbf{For each of the following statements, write if it is ``True,'' ``False,'' or ``We
    don't know.''} Justify your answers.\\
    \begin{enumerate}[(a)]
        \item A problem $Q$ can be in $\mathcal{P}$ and $\mathcal{NP}$.\\
            \bigskip
            \textbf{True}. According to the definitions given in \cite{modernapp}, $\mathcal{P} =
            \bigcup_{c \geq 1}$DTIME$(n^c)$, the class of languages decidable in polynomial time by
            a \emph{deterministic} Turing machine and $\mathcal{NP} = \bigcup_{c \in
            \mathbb{N}}$NTIME$(n^c)$, the class of languages decidable in polynomial time by a
            \emph{nondeterministic} Turing machine. Further, it is shown that $\mathcal{P} \subseteq
            \mathcal{NP}$. Therefore, if $Q \in \mathcal{P}$, then $Q \in \mathcal{NP}$. This also
            follows from the intuitive notion that $\mathcal{P}$ is the class of problems with
            ``efficiently computable solutions'' and $\mathcal{NP}$ is the class of problems with
            ``efficiently verifiable solutions'', i.e., any solutions that are efficiently solvable
            must necessarily also be efficiently verifiable. The reverse set membership question,
            whether $\mathcal{NP} \subseteq \mathcal{P}$, is not currently known to be true or
            false.
        \item A problem $Q$ can be in $\mathcal{P}$ and $\mathcal{NP}$-Complete.\\
            \bigskip
            \textbf{We don't know}. According to the definitions given in \cite{modernapp}, a
            language $L'$ is $\mathcal{NP}$-Complete if $L'$ is $\mathcal{NP}$-Hard and $L' \in
            \mathcal{NP}$, and that a language $L'$ is $\mathcal{NP}$-Hard if $L$ is polynomial-time
            Karp reducible to $L'$ for all $L \in \mathcal{NP}$. It has not been shown for any $L$
            that $L \in \mathcal{P}$ and $L$ is $\mathcal{NP}$-Hard. If this were to be shown, it
            would give us an affirmative answer to the question of whether $\mathcal{P} =
            \mathcal{NP}$ (theorem 2.8 in \cite{modernapp}).
        \item A problem $Q$ can be in $\mathcal{NP}$ and $\mathcal{NP}$-Complete.\\
            \bigskip
            \textbf{True}. This is directly from the definition of $\mathcal{NP}$-Complete: ``We say
            that $L'$ is $\mathcal{NP}$-Complete if $L'$ is $\mathcal{NP}$-Hard and $L' \in
            \mathcal{NP}$'' (from definition 2.7 in \cite{modernapp}). If $Q$ is in
            $\mathcal{NP}$-Complete, then $Q$ is necessarily in $\mathcal{NP}$.
    \end{enumerate}
    %%%%%%%%%%%%%%%%%%%%%%%%%%%%%%%%%%%%%%%%%%%%%%%%%%%%%%%%%%%%%%%%%%%%%%%%%%%%%%%%%%%%%%%%%%%%%%%%
    % Problem 2.
    %%%%%%%%%%%%%%%%%%%%%%%%%%%%%%%%%%%%%%%%%%%%%%%%%%%%%%%%%%%%%%%%%%%%%%%%%%%%%%%%%%%%%%%%%%%%%%%%
    2. Let $X$ be a problem that is in the class $\mathcal{NP}$. \textbf{For each of the following
    statements, write if it is ``True,'' ``False,'' or ``We don't know.''} Justify your answers.\\
    \begin{enumerate}[(a)]
        \item Solutions to $X$ can be verified with a polynomial-time algorithm.\\
            \bigskip
            \textbf{True}. This is directly from the definition of $\mathcal{NP}$ (definition 2.1 in
            \cite{modernapp}); stated informally, we can say a language $L$ is in $\mathcal{NP}$ if
            membership in $L$ can be verified efficiently (i.e., in polynomial time).
        \item There is no polynomial-time algorithm for $X$.\\
            \bigskip
            \textbf{We don't know}. If it was known that $X \in \mathcal{P} \subseteq \mathcal{NP}$,
            then the claim would have been false (i.e., it would give us confidence that such an
            algorithm exists), but for the more general case of $X \in \mathcal{NP}$, the answer is
            still uncertain (until and unless it is also proven that $\mathcal{NP} \subseteq
            \mathcal{P}$).
        \item If there exists a polynomial-time algorithm for $X$, then $\mathcal{P} =
            \mathcal{NP}$.\\
            \bigskip
            \textbf{False} (assuming $X$ is not also $\mathcal{NP}$-Hard). If $X$ were also
            $\mathcal{NP}$-Hard (such that all problems in $\mathcal{NP}$ were polynomial-time Karp
            reducible to $X$), the claim would have been true (definition 2.7 and theorem 2.8 in
            \cite{modernapp}). Under this additional assumption of $X$ being both in $\mathcal{NP}$
            and $\mathcal{NP}$-Hard, $X$ would also fit the description (as stated above) of being
            $\mathcal{NP}$-Complete. Since having a polynomial-time algorithm for $X$ implies $X \in
            \mathcal{P}$, $\mathcal{P} = \mathcal{NP}$ must be true according to theorem 2.8: ``If
            language $L$ is $\mathcal{NP}$-Complete, then $L \in \mathcal{P}$ \emph{if and only if}
            $\mathcal{P} = \mathcal{NP}$''. The biconditional tells us that we can assume the claim
            is false if $X$ is not $\mathcal{NP}$-Complete. If it is not known whether $X$ is
            $\mathcal{NP}$-Hard or not, then the claim becomes uncertain.
        \item If there exists a polynomial-time algorithm for $X$, then there must exist a
            polynomial-time algorithm for the Traveling Salesman Problem.\\
            \bigskip
            \textbf{False} (assuming $X$ is not also $\mathcal{NP}$-Hard). Using essentially the
            same reasoning as the question directly above, we can say that since $X$ is known to be
            in $\mathcal{NP}$, $X$ would also need to be $\mathcal{NP}$-Hard so that any problem in
            $\mathcal{NP}$ (such as the Traveling Salesman Problem) is proven to be polynomial-time
            Karp reducible to $X$. Being in both $\mathcal{NP}$ and $\mathcal{NP}$-Hard would imply
            $X$ is $\mathcal{NP}$-Complete. Hence, again by theorem 2.8 (quoted in part above), we
            can say that $X$ not being $\mathcal{NP}$-Complete implies that the claim is false. Also
            as above, if $X$'s $\mathcal{NP}$-Hardness is uncertain, then the claim is uncertain as
            well.
        \item If there exists a polynomial-time algorithm for the Traveling Salesman Problem, then
            there must exist a polynomial-time algorithm for $X$.\\
            \bigskip
            \textbf{True}. The Traveling Salesman Problem (TSP) is known to be
            $\mathcal{NP}$-Complete according to \cite{modernapp}. Since $\mathcal{NP}$-Completeness
            implies the problem is both in $\mathcal{NP}$ and $\mathcal{NP}$-Hard, all problems in
            $\mathcal{NP}$ must be polynomial-time Karp reducible to the TSP (from definition 2.7
            partially quoted above). In other words, the TSP would be ``at least as hard'' as $X$
            and TSP $\in \mathcal{P}$ implies $X \in \mathcal{P}$.
    \end{enumerate}

    \bibliography{complexity}
    \bibliographystyle{plain}
\end{flushleft}
\end{document}

% EOF.
