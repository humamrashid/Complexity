% Midterm.
% Humam Rashid
% Spring 2020, CISC 7214X, Prof. Cox

\documentclass{scrartcl}
\usepackage{amsmath, amsthm, amssymb}
\usepackage[a4paper, margin=1in]{geometry}
\usepackage[shortlabels]{enumitem}
\begin{document}
\begin{flushleft}
    Humam Rashid\\
    Midterm\\
    CISC 7214X, Prof. Cox\\
\end{flushleft}
    \bigskip
    
    %%%%%%%%%%%%%%%%%%%%%%%%%%%%%%%%%%%%%%%%%%%%%%%%%%%%%%%%%%%%%%%%%%%%%%%%%%%%%%%%%%%%%%%%%%%%%%%%
    % Problem 1 on proof by induction.
    %%%%%%%%%%%%%%%%%%%%%%%%%%%%%%%%%%%%%%%%%%%%%%%%%%%%%%%%%%%%%%%%%%%%%%%%%%%%%%%%%%%%%%%%%%%%%%%%
    \noindent
    1. Solve the summation.
    \begin{align*}
        \sum_{i=1}^{n} 4i
    \end{align*}
    \textit{Solution}:
    \begin{align*}
        \sum_{i=1}^{n} 4i &= 4 \sum_{i=1}^{n} i\\
                          &= 4 \Bigg(\frac{n (n + 1)}{2}\Bigg)\;\text{by sum of first $n$ natural
                          numbers}\\
                          &= 2n(n + 1)
    \end{align*}

    %%%%%%%%%%%%%%%%%%%%%%%%%%%%%%%%%%%%%%%%%%%%%%%%%%%%%%%%%%%%%%%%%%%%%%%%%%%%%%%%%%%%%%%%%%%%%%%%
    % Problem 2 on recurrence equations.
    %%%%%%%%%%%%%%%%%%%%%%%%%%%%%%%%%%%%%%%%%%%%%%%%%%%%%%%%%%%%%%%%%%%%%%%%%%%%%%%%%%%%%%%%%%%%%%%%
    \noindent
    2. Solve the following recurrence and prove the solution correct with induction:
    \begin{align*}
        T(1) &= 0\\
        T(n) &= n + 2 T(\frac{n}{2})
    \end{align*}
    \textit{Solution}: $T(n) = n\;lg(n)$, by guessing.
    \medskip
    \begin{proof}
        For $k < n$, we have:
        \begin{align*}
            T(k) &= k\;lg(k)\\
            T(n) &= n + 2 T(\frac{n}{2})\\
                 &= n + 2 (\frac{n}{2}\;lg(n) - \frac{n}{2})\\
                 &= n + n\;lg(n) - n\\
                 &= n\;lg(n)
        \end{align*}
    \end{proof}
    %%%%%%%%%%%%%%%%%%%%%%%%%%%%%%%%%%%%%%%%%%%%%%%%%%%%%%%%%%%%%%%%%%%%%%%%%%%%%%%%%%%%%%%%%%%%%%%%
    % Problem 3 on data structure for insertion and max extraction for set of integers.
    %%%%%%%%%%%%%%%%%%%%%%%%%%%%%%%%%%%%%%%%%%%%%%%%%%%%%%%%%%%%%%%%%%%%%%%%%%%%%%%%%%%%%%%%%%%%%%%%
    \noindent
    3. Suppose you want to be able to perform the following 2 operations on a set of integers.
    Insert a new item and remove the maximum item. Describe a data structure that supports this and
    analyze its running time.\\

    \noindent
    \textit{Solution}: One data structure that readily supports the given requirements is a
    \emph{Max-Priority Queue}. A Max-Priority Queue can be built on top of a Max-Heap. Building the
    Max-Heap from a set of $n$ unsorted integers takes $O(n)$ time. Removing the maximum item is an
    $O(lg\,n)$ operation since returning the maximum can be done in constant time but maintaining
    the Max-Heap takes $O(lg\,n)$ time. Inserting an item into a Max-Heap with $n$ items also takes
    $O(lg\,n)$ because of the Max-Heap maintenance procedure. (I am using Max-Priority Queue here as
    defined in Cormen et al., effectively making it a simple abstraction over a Max-Heap).

    \newpage
    %%%%%%%%%%%%%%%%%%%%%%%%%%%%%%%%%%%%%%%%%%%%%%%%%%%%%%%%%%%%%%%%%%%%%%%%%%%%%%%%%%%%%%%%%%%%%%%%
    % Problem 4 on data structure on algorithm for finding shortest-path tree in a network.
    %%%%%%%%%%%%%%%%%%%%%%%%%%%%%%%%%%%%%%%%%%%%%%%%%%%%%%%%%%%%%%%%%%%%%%%%%%%%%%%%%%%%%%%%%%%%%%%%
    \noindent
    4. Suppose you are given a network with $n$ cities, and a direct connection between each pair of
    cities with the distance between them. Given a starting city $s$, describe an algorithm to find
    the shortest path from $s$ to each city in the network. Analyze the complexity.\\

    \noindent
    \textit{Solution}: An effective algorithm to use here is \emph{Dijkstra's Shortest Path
    Algorithm}. This algorithm can be use to construct a shortest-path tree from the network of $n$
    cities, using $s$ as the starting node. The time complexity of Dijkstra's shortest path
    algorithm is dependent on the data structure used to implement the procedure that gives us the
    shortest-path estimate for a vertex $u \in V - S$, where $V$ is the set of vertices representing
    the cities (i.e., $|V| = n$) and $S$ is the set of vertices for which the final shortest-path
    weights from $s$ are already determined. Given a procedure \verb|EXTRACT-MIN()| to get $u$, the
    following describes several well known methods of implementing the procedure and the resulting
    overall time complexity of Dijktra's shortest path algorithm ($E$ is the set of edges in the
    network):
    \begin{enumerate}[(a)]
        \item If \verb|EXTRACT-MIN| is implemented using a simple linked-list or array requiring
            linear search to find the minimum, Dijkstra's shortest path algorithm has time
            complexity: $O(|V|^2)$ or $O(n^2)$.
        \item If \verb|EXTRACT-MIN| is implemented using a Binary-Min Heap, then Dijkstra's shortest
            path algorithm has time complexity: $O((|V| + |E|)\;lg\,|V|)$ or $O((n + |E|)\;lg\,n)$.
            For connected networks (as in our case), this further reduces to $O(|E|\;lg\,|V|)$ or
            $O(|E|\;lg\,n)$.
        \item If \verb|EXTRACT-MIN| is implemented using a Fibonacci Heap, then Dijkstra's shortest
            path algorithm has time complexity: $O(|V|\;lg\,|V| + |E|)$ or $O(n\;lg\,n + |E|)$.
    \end{enumerate}

\end{document}

% EOF.
