% Midterm.
% Humam Rashid
% Spring 2020, CISC 7214X, Prof. Cox

\documentclass{scrartcl}
\usepackage{amsmath, amsthm, amssymb}
\usepackage[a4paper, margin=1in]{geometry}
\begin{document}
\begin{flushleft}

    Humam Rashid\\
    Midterm\\
    CISC 7214X, Prof. Cox\\
    \bigskip
    
    %%%%%%%%%%%%%%%%%%%%%%%%%%%%%%%%%%%%%%%%%%%%%%%%%%%%%%%%%%%%%%%%%%%%%%%%%%%%%%%%%%%%%%%%%%%%%%%%
    % Problem 1 on proof by induction.
    %%%%%%%%%%%%%%%%%%%%%%%%%%%%%%%%%%%%%%%%%%%%%%%%%%%%%%%%%%%%%%%%%%%%%%%%%%%%%%%%%%%%%%%%%%%%%%%%
    1. Solve the summation.
    \begin{align*}
        \sum_{i=1}^{n} 4i
    \end{align*}
    \textit{Solution}:
    \begin{align*}
        \sum_{i=1}^{n} 4i &= 4 \sum_{i=1}^{n} i\\
                          &= 4 (\frac{n (n + 1)}{2})\;\text{by sum of first $n$ natural numbers}\\
                          &= 2n(n + 1)
    \end{align*}
    %%%%%%%%%%%%%%%%%%%%%%%%%%%%%%%%%%%%%%%%%%%%%%%%%%%%%%%%%%%%%%%%%%%%%%%%%%%%%%%%%%%%%%%%%%%%%%%%
    % Problem 2 on recurrence equations.
    %%%%%%%%%%%%%%%%%%%%%%%%%%%%%%%%%%%%%%%%%%%%%%%%%%%%%%%%%%%%%%%%%%%%%%%%%%%%%%%%%%%%%%%%%%%%%%%%
    2. Solve the following recurrence and prove the solution correct with induction:
    \begin{align*}
        T(1) &= 0\\
        T(n) &= n + 2 T(\frac{n}{2})
    \end{align*}
    \textit{Solution}: $T(n) = n lg(n)$, by guessing.
    \medskip
    \begin{proof}
        For $k < n$, we have:\\
        \begin{align*}
            T(k) &= k lg(k)\\
            T(n) &= n + 2 T(\frac{n}{2})\\
                 &= n + 2 (\frac{n}{2} lg(n) - \frac{n}{2})\\
                 &= n + n lg(n) - n\\
                 &= n lg(n)
        \end{align*}
    \end{proof}
    %%%%%%%%%%%%%%%%%%%%%%%%%%%%%%%%%%%%%%%%%%%%%%%%%%%%%%%%%%%%%%%%%%%%%%%%%%%%%%%%%%%%%%%%%%%%%%%%
    % Problem 3 on data structure for insertion and max extraction for integers.
    %%%%%%%%%%%%%%%%%%%%%%%%%%%%%%%%%%%%%%%%%%%%%%%%%%%%%%%%%%%%%%%%%%%%%%%%%%%%%%%%%%%%%%%%%%%%%%%%
    3. Suppose you want to be able to perform the following 2 operations on a set of integers.
    Insert a new item and remove the maximum item. Describe a data structure that supports this and
    analyze its running time.\\
    \bigskip
    \textit{Solution}: One data structure that readily supports the given requirements is a
    Max-Priority Queue. A Max-Priority Queue can be built on top of a Max-Heap. Building the
    Max-Heap from a set of $n$ unsorted integers takes $O(n)$ time. Removing the maximum item is an
    $O(lg(n))$ operation since returning the maximum can be done in constant time but maintaining
    the Max-Heap takes $O(lg(n))$ time. Inserting an item into a Max-Heap with $n$ items also takes
    $O(lg(n))$ because of the Max-Heap maintenance procedure. (I am using Max-Priority Queue here as
    defined in Cormen et al., effectively making it a simple abstraction over a Max-Heap).

\end{flushleft}
\end{document}

% EOF.
