% Midterm.
% Humam Rashid
% Spring 2020, CISC 7214X, Prof. Cox

\documentclass{scrartcl}
\usepackage{amsmath, amsthm, amssymb}
\usepackage[a4paper, margin=1in]{geometry}
\usepackage[shortlabels]{enumitem}
\begin{document}
\begin{flushleft}
    Humam Rashid\\
    Midterm, Spring 2020\\
    CISC 7214X, Prof. Cox
\end{flushleft}
    \bigskip
    
    %%%%%%%%%%%%%%%%%%%%%%%%%%%%%%%%%%%%%%%%%%%%%%%%%%%%%%%%%%%%%%%%%%%%%%%%%%%%%%%%%%%%%%%%%%%%%%%%
    % Problem 1 on solving summations.
    %%%%%%%%%%%%%%%%%%%%%%%%%%%%%%%%%%%%%%%%%%%%%%%%%%%%%%%%%%%%%%%%%%%%%%%%%%%%%%%%%%%%%%%%%%%%%%%%
    \noindent
    1. Solve the summation.
    \begin{align*}
        \sum_{i=1}^{n} 4i
    \end{align*}
    \textit{Solution}:
    \begin{align*}
        \sum_{i=1}^{n} 4i &= 4 \sum_{i=1}^{n} i\\[6pt]&\text{by distributive identity}\\[6pt]
                          &= 4 \bigg(\frac{n (n + 1)}{2}\bigg)\\[6pt]&\text{by sum of first $n$
                          natural numbers}\\[6pt]
                          &= 2n(n + 1)
    \end{align*}

    %%%%%%%%%%%%%%%%%%%%%%%%%%%%%%%%%%%%%%%%%%%%%%%%%%%%%%%%%%%%%%%%%%%%%%%%%%%%%%%%%%%%%%%%%%%%%%%%
    % Problem 2 on solving and proving recurrence equations.
    %%%%%%%%%%%%%%%%%%%%%%%%%%%%%%%%%%%%%%%%%%%%%%%%%%%%%%%%%%%%%%%%%%%%%%%%%%%%%%%%%%%%%%%%%%%%%%%%
    \bigskip
    \noindent
    2. Solve the following recurrence and prove the solution correct with induction:
    \begin{align*}
        T(1) &= 0\\
        T(n) &= n + 2 T\bigg(\frac{n}{2}\bigg)
    \end{align*}
    \textit{Solution}: $T(n) = n\;lg(n)$, claimed. We see that $T(1) = 1\;lg(1) = 0$.\\
    \begin{proof}
        For $k < n$, we have:
        \begin{align*}
            T(k) &= k\;lg(k)\\
            T(n) &= n + 2 T\bigg(\frac{n}{2}\bigg)\\
                 &= n + 2 \bigg(\frac{n}{2}\;lg(n) - \frac{n}{2}\bigg)\\
                 &= n + n\;lg(n) - n\\
                 &= n\;lg(n)
        \end{align*}
    \end{proof}
    %%%%%%%%%%%%%%%%%%%%%%%%%%%%%%%%%%%%%%%%%%%%%%%%%%%%%%%%%%%%%%%%%%%%%%%%%%%%%%%%%%%%%%%%%%%%%%%%
    % Problem 3 on data structure for insertion and max extraction for set of integers.
    %%%%%%%%%%%%%%%%%%%%%%%%%%%%%%%%%%%%%%%%%%%%%%%%%%%%%%%%%%%%%%%%%%%%%%%%%%%%%%%%%%%%%%%%%%%%%%%%
    \newpage
    \noindent
    3. Suppose you want to be able to perform the following 2 operations on a set of integers.
    Insert a new item and remove the maximum item. Describe a data structure that supports this and
    analyze its running time.\\

    \noindent
    \textit{Answer}: One data structure that readily supports the given requirements is a
    \emph{Max-Priority Queue}. A Max-Priority Queue can be built on top of a Max-Heap. Building the
    Max-Heap from a set of $n$ unsorted integers takes $O(n)$ time. Removing the maximum item is an
    $O(lg\,n)$ operation since returning the maximum can be done in constant time but maintaining
    the Max-Heap takes $O(lg\,n)$ time. Inserting an item into a Max-Heap with $n$ items also takes
    $O(lg\,n)$ because of the Max-Heap maintenance procedure. (I am using Max-Priority Queue here as
    defined in Cormen et al., effectively making it a simple abstraction over a Max-Heap).\\

    %%%%%%%%%%%%%%%%%%%%%%%%%%%%%%%%%%%%%%%%%%%%%%%%%%%%%%%%%%%%%%%%%%%%%%%%%%%%%%%%%%%%%%%%%%%%%%%%
    % Problem 4 on algorithm for finding shortest-path tree in a network.
    %%%%%%%%%%%%%%%%%%%%%%%%%%%%%%%%%%%%%%%%%%%%%%%%%%%%%%%%%%%%%%%%%%%%%%%%%%%%%%%%%%%%%%%%%%%%%%%%
    \bigskip
    \noindent
    4. Suppose you are given a network with $n$ cities, and a direct connection between each pair of
    cities with the distance between them. Given a starting city $s$, describe an algorithm to find
    the shortest path from $s$ to each city in the network. Analyze the complexity.\\

    \noindent
    \textit{Answer}: An effective algorithm to use here is \emph{Dijkstra's Shortest Path
    Algorithm}. This algorithm can be used to construct a shortest-path tree from the network of $n$
    cities, using $s$ as the starting node. The time complexity of Dijkstra's shortest path
    algorithm is dependent on the data structure used to implement the procedure that gives us the
    shortest-path estimate for a vertex $u \in V - S$, where $V$ is the set of vertices representing
    the cities (i.e., $|V| = n$) and $S$ is the set of vertices for which the final shortest-path
    weights from $s$ are already determined. Given a procedure \verb|EXTRACT-MIN()| to get $u$, the
    following describes several well known methods of implementing the procedure and the resulting
    overall time complexity of Dijkstra's shortest path algorithm ($E$ is the set of edges in the
    network):
    \begin{enumerate}[(a)]
        \item If \verb|EXTRACT-MIN| is implemented using a simple linked-list or array requiring
            linear search to find the minimum, Dijkstra's shortest path algorithm has time
            complexity: $O(|V|^2)$ or $O(n^2)$.\\
        \item If \verb|EXTRACT-MIN| is implemented using a Binary-Min Heap, then Dijkstra's shortest
            path algorithm has time complexity: $O((|V| + |E|)\;lg\,|V|)$ or $O((n + |E|)\;lg\,n)$.
            For connected networks (as in our case), this further reduces to $O(|E|\;lg\,|V|)$ or
            $O(|E|\;lg\,n)$.\\
        \item If \verb|EXTRACT-MIN| is implemented using a Fibonacci Heap, then Dijkstra's shortest
            path algorithm has time complexity: $O(|V|\;lg\,|V| + |E|)$ or $O(n\;lg\,n + |E|)$.
    \end{enumerate}

    %%%%%%%%%%%%%%%%%%%%%%%%%%%%%%%%%%%%%%%%%%%%%%%%%%%%%%%%%%%%%%%%%%%%%%%%%%%%%%%%%%%%%%%%%%%%%%%%
    % Problem 6 on complexity classes.
    %%%%%%%%%%%%%%%%%%%%%%%%%%%%%%%%%%%%%%%%%%%%%%%%%%%%%%%%%%%%%%%%%%%%%%%%%%%%%%%%%%%%%%%%%%%%%%%%
    \bigskip
    \noindent
    6. Show that \textsf{NL} $\subseteq \mathcal{P}$.\\

    %%%%%%%%%%%%%%%%%%%%%%%%%%%%%%%%%%%%%%%%%%%%%%%%%%%%%%%%%%%%%%%%%%%%%%%%%%%%%%%%%%%%%%%%%%%%%%%%
    % Problem 7 on complexity classes.
    %%%%%%%%%%%%%%%%%%%%%%%%%%%%%%%%%%%%%%%%%%%%%%%%%%%%%%%%%%%%%%%%%%%%%%%%%%%%%%%%%%%%%%%%%%%%%%%%
    \bigskip
    \noindent
    7. How do we know that \textsf{NPSPACE $=$ PSPACE}?\\

    \noindent
    \textit{Answer}: This is a direct result of Savitch's theorem, which says \textsf{NSPACE$(s(n))
    \subseteq$ SPACE$(s(n)^2)$}. That is, for space complexity, any nondeterministic Turing machine
    that uses $s(n)$ space can be converted to a deterministic Turing machine that uses only
    $s(n)^2$ space. \textsf{NPSPACE $=$ PSPACE} is a direct result based on the fact that the square
    of any polynomial is still a polynomial.\\

    %%%%%%%%%%%%%%%%%%%%%%%%%%%%%%%%%%%%%%%%%%%%%%%%%%%%%%%%%%%%%%%%%%%%%%%%%%%%%%%%%%%%%%%%%%%%%%%%
    % Problem 8 on complexity classes.
    %%%%%%%%%%%%%%%%%%%%%%%%%%%%%%%%%%%%%%%%%%%%%%%%%%%%%%%%%%%%%%%%%%%%%%%%%%%%%%%%%%%%%%%%%%%%%%%%
    \newpage
    \noindent
    8. Let $X$ be a problem that is in the class $\mathcal{NP}$. For each of the following, write if
    it is ``True,'' ``False,'' or ``We don't know.'' Justify your answer.
    \begin{enumerate}[(a)]
        \item Solutions to $X$ can be verified with a polynomial-time algorithm.\\

            \textit{True}. This is directly from the definition of $\mathcal{NP}$.\\
        \item There is no polynomial-time algorithm for $X$.\\

            \textit{We don't know}. If it was known that $X \in \mathcal{P} \subseteq \mathcal{NP}$,
            then the claim would have been false (i.e., it would give us confidence that such an
            algorithm exists), but for the more general case of $X \in \mathcal{NP}$, the answer is
            still uncertain (until and unless it is also proven that $\mathcal{NP} \subseteq
            \mathcal{P} \Rightarrow \mathcal{P} = \mathcal{NP}$).\\
        \item If $X$ is $\mathcal{NP}$-Hard, then $X$ is $\mathcal{NP}$-Complete.\\

            \textit{True}. An $\mathcal{NP}$-Complete problem is one that is $\mathcal{NP}$-Hard and
            also in $\mathcal{NP}$, therefore this is directly from the definition of
            $\mathcal{NP}$-Complete.\\
        \item If there exists a polynomial-time algorithm for \textsf{SAT}, then there exists a
            polynomial-time algorithm for $X$.\\

            \textit{True}. The \textsf{SAT} problem is known to be $\mathcal{NP}$-Complete. Since
            $\mathcal{NP}$-Completeness implies the problem is both in $\mathcal{NP}$ and
            $\mathcal{NP}$-Hard, all problems in $\mathcal{NP}$ must be polynomial-time Karp
            reducible to \textsf{SAT} (by definition of $\mathcal{NP}$-Hard). In other words,
            \textsf{SAT} is ``at least as hard'' as $X$ and therefore \textsf{SAT} $\in \mathcal{P}
            \Rightarrow X \in \mathcal{P}$.
    \end{enumerate}

\end{document}

% EOF.
